\begin{lecture}{4}{Uniform Hardness vs.\ Randomness}{February 10, 2022}

\subsection*{The Upshot}

\begin{enumerate}
  \item Distinguishers for the NW-PRG with hard function $f$ imply learning
    algorithms for $f$.
  \item We can significantly boost the learning algorithm for functions that
    are downward self-reducible and whose truthtable corresponds to a LLDC.
\end{enumerate}


\subsection{Uniform Circuits}

\begin{definition}[$\eps$-PRG for uniform circuits]
	An algorithm $G$ is an $\eps$-PRG for uniform circuits generated in time
	$t(n)$ if for every turing machine $D$ with running time $t(n)$ and
	large enough $n \in \N$
	\[
		\Pr\brac{D(1^n) \text{ prints an } \eps\text{-distinguisher circuit }
		D_n \text{ for } G(1^n, u_{l(n)}} \leq \eps.
	\]
\end{definition}

\begin{definition}
	Let $\overline{x} = \set{\overline{x}_n}$ be a sequence of distributions
	over $\set{0, 1}^n$. We say that $\overline{x}$ is $p$-time samplable
	if there is a probabilistic machine $S$ with runtime $p(n)$ such that
	$S(1^n)$ has the distribution $\overline{x}(n)$.
\end{definition}

\begin{theorem}
	Assume that for all polynomials $p(n)$ there is an $\eps$-PRG $G$ for
	uniform circuits generated in time $p(n)$, such that $G$ has polynomial
	running time, and uses a $\log$-size seed. Then for all
	$L \in \BPP$ and poly-time samplable distributions $\overline{x}$,%
	\footnote{Think of $\overline{x}$ as an adversary (limited to polynomial
	time) trying to generate hard inputs with $\geq \eps$ probability.}
	there exists a language $L' \in \textsf{P}$ such that:
	\[
		\Pr_{x \sim \overline{x}}\brac{L(x) \neq L'(x)} \leq \eps.
	\]
\end{theorem}

\begin{theorem}
	If
	$\SPACE\brac{O(n)}$ is hard for $\BPTIME\brac{2^{\eps n}}$
	then
	$\RP = \P$ ``on average''.
\end{theorem}

\newcommand{\fws}{\ensuremath{ f^{\text{ws}} }}
\begin{theorem}
	There exists a language $\fws \in \P^{\sharpp}$ such that if $\fws$
	is hard for $\BPTIME\brac{2^{\eps n}}$ then
	$\BPP = \P$ ``on average''.
\end{theorem}

\subsection{Learning via Uniform Distinguishers}

\subsection{Boosting via Downward Self-Reducibility and LLDCs}

\end{lecture}
