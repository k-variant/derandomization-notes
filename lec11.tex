\chapter{Derandomization Requires Circuit Lower Bounds in NTIME}\label{lec:11}

\section[\texorpdfstring{The Class $\MA$}{The Class MA}]{The Class $\MA$}

\begin{definition}[$\MA$]
  $\MA$ is the complexity class of all languages $L$ for which there is a probabilistic 
  polytime verifier $V$ and polynomials $p,r$ such that
  \begin{align*}
      x \in L \implies &
      \exists z \in \set{0,1}^{p(|x|)} \Pr_{y \in \set{0,1}^{r(|x|)}} \brac{V(x,y,z) = 1} \ge \frac{2}{3}
      \\
      x \not\in L \implies &
      \forall z \in \set{0,1}^{p(|x|)} \Pr_{y \in \set{0,1}^{r(|x|)}} \brac{V(x,y,z) = 0} \ge \frac{2}{3}.
  \end{align*}
\end{definition}

One can think of $\MA$ in terms of languages that can be decided by Arthur through a one 
message interaction with Merlin, where Merlin has unbounded computation and Arthur is 
restricted to polytime computation with randomness:
If $x \in L$, then there is some proof of membership that Merlin can send that Arthur is 
likely to accept.
On the other hand, if $x \not\in L$, then there is no proof of membership that Merlin can 
send that's likely to trick Arthur into accepting.

Succinctly, one can think of $\MA$ as a randomized version of $\NP$.
Put otherwise, $\MA$ is to $\NP$ as $\BPP$ is to $\P$.
Just as we wish to answer the $\BPP$ vs $\P$ question, this perspective begets the $\MA$ vs
$\NP$ question; is $\MA = \NP$?

$\MA$ vs $\NP$ is a question we do not yet know the answer to.
This does not stop us from asking:
How believable is $\MA = \NP$?
It is at least as believable as $\BPP = \P$, and a proof of $\MA = \NP$ may even come with
less blood, sweat, and tears.
\begin{lemma}\label{lem:manp-ez}
  $\BPP = \P \implies \MA = \NP$
\end{lemma}
\begin{proofsk}
  Once non-determinism has been used up (Merlin has sent Arthur a proof), we are left with
  a verification problem in $\BPP$ which can be derandomized to $\P$.
\end{proofsk}

\section[\texorpdfstring{Derandomizations of $\MA$}{Derandomizations of MA}]{Derandomizations of $\MA$}

$\MA = \NP$ is the strongest derandomization of $\MA$ (up to a polynomial slowdown).
We know that $\NP \subseteq \MA \subseteq \NEXP$ and so we can try to say something easier
about the extent to which $\MA$ can be derandomized.
For example, $\MA \subsetneq \NEXP$ is already a non-trivial statement to make, but
it is certainly ``easier'' than $\MA = \NP$.

Constrast this with derandomizations of $\BPP$.
Weakening the hardness of a function given to the NW-generator yields a PRG that requires 
longer random seeds and thus gives weaker derandomizations.

In the remaining sections we will show a conditional derandomization that is much weaker 
than $\MA = \NP$.
In a sense, it will be the weakest non-trivial derandomization we show in these notes;
lemma~\ref{lem:manp-ez} suggests, qualitatively, that derandomizations of $\MA$ are 
easier endeavors than similar derandomizations of $\BPP$ and, within the scope of 
derandomizing $\MA$, we will aim to show something that is weak albeit non-trivial.

\section{Hardness vs.\ Randomness for Proof Systems}

\section{Easy Witnesses}
