\chapter{Hardness Amplification via Error-Correcting Codes}
\label{lec:03}

In \Cref{lec:02}, we saw that if there is a language $L$ in $\E{}$ that cannot
be computed on average by a circuit of size $2^{\eps n}$, then $\BPP = \P$. In
this lecture, we will show---through the magic of polynomials---that we can
make a signficantly weaker assumption and obtain the same conclusion: If there
is a language in $\E$ which is not computable by any circuit of size $2^{\eps
n}$, then $\BPP = \P$.

\section{A Primer on Coding Theory}

\section{Hardness Amplification via Locally List Decodable Codes}

\section{Reed-Solomon Codes}

\section{Reed-Muller Codes}
