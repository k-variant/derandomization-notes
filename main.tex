\documentclass[12pt]{article}

\usepackage{graphicx}
\usepackage{amsmath,amssymb}
\usepackage{amsthm}
\usepackage{xspace}
\usepackage{centernot}
\usepackage{enumitem}
\usepackage[sans]{dsfont}
\usepackage{biblatex}
\usepackage[immediate]{silence}
\usepackage{nicefrac}


% These come from sectsty; doesn't affect us.
\WarningFilter[sectsty]{latex}{Command}
\usepackage{sectsty}
\DeactivateWarningFilters[sectsty]

\usepackage{tocloft}

\usepackage[usenames,dvipsnames]{xcolor}
\definecolor{DarkRed}{rgb}{0.5,0.1,0.1}
\definecolor{ForestGreen}{rgb}{0.1333,0.5451,0.1333}
\usepackage{hyperref}
\hypersetup{
  linktocpage=true,
  citecolor=ForestGreen,
  colorlinks=true,
  linkcolor=DarkRed,
  linktoc=all,
}

% A better ref system
\usepackage[nameinlink, capitalise]{cleveref}

% Fonts
\usepackage[T1]{fontenc}
\usepackage{libertine}
\usepackage[libertine]{newtxmath}
\usepackage{roboto}
\usepackage{inconsolata}
\usepackage[bb=esstix]{mathalpha}

% Complexity classes
\usepackage{complexity}

\newenvironment{proofsk}{%
  \renewcommand{\proofname}{Proof sketch}\proof}{\endproof}

\newtheoremstyle{sansserif}%
  {10pt}%                Space above
  {10pt}%                Space below
  {\itshape}%            Body font
  {}%                    Indent amount
  {\normalfont\sffamily\small}%  Theorem head font
  {\textnormal{.}}%      Punctuation after theorem head
  {.5em}%                Space after theorem head
  {}%                    Theorem head spec

\newtheorem{theorem}{Theorem}[section]
\newtheorem{claim}[theorem]{Claim}
\newtheorem{conjecture}[theorem]{Conjecture}
\newtheorem{definition}[theorem]{Definition}
\newtheorem{corollary}[theorem]{Corollary}
\newtheorem{exercise}[theorem]{Exercise}
\newtheorem{lemma}[theorem]{Lemma}
\newtheorem{observation}[theorem]{Observation}
\newtheorem{proposition}[theorem]{Proposition}
\newtheorem{tool}[theorem]{Tool}

\setlength{\topmargin}{0pt}
\setlength{\textheight}{9in}
\setlength{\headheight}{0pt}
\setlength{\headsep}{0pt}
\setlength{\oddsidemargin}{0.25in}
\setlength{\textwidth}{6in}

\pagestyle{plain}

% Hopefully this doesn't come back to bite me...
\WarningFilter{latex}{Empty bibliography}

\crefname{lecture}{lecture}{lectures}
\Crefname{lecture}{Lecture}{Lectures}


\usepackage{mathtools}

\def\eps{\varepsilon}
\def\Exp{\mathbb{E}}
\def\N{\mathbb{N}}
\def\R{\mathbb{R}}

\def\sharpp{\textsf{\#P}\xspace}

\DeclareMathOperator{\Bern}{Bern}
\DeclarePairedDelimiter\brac{[}{]}
\DeclarePairedDelimiter\card{\lvert}{\rvert}
\DeclarePairedDelimiter\set{\{}{\}}
\DeclarePairedDelimiter\floor{\lfloor}{\rfloor}
\DeclarePairedDelimiter\paren{(}{)}

\newcommand{\comment}[1]{\textcolor{red}{[\textbf{#1}]}}

\newlang{\CAPP}{CAPP}
\newlang{\PERM}{PERM}

\renewclass{\promiseBPP}{prBPP}
\newclass{\prBPP}{prBPP}
\newclass{\prBPTIME}{prBPTIME}
\newclass{\prRP}{prRP}
\newclass{\prP}{prP}

\renewcommand{\Pr}{\mathbb{P}}
\addbibresource{main.bib}

\title{\bf Derandomization and Its Connections Throughout Complexity Theory}
\author{}

\begin{document}
  \allsectionsfont{\sffamily}

  \newcommand{\pr}[1]{\textsf{pr#1}}
  \newcommand{\dtime}[1]{\DTIME\brac*[#1]}

  \maketitle

  % Table of contents with "Contents" line.
  \makeatletter
  \@starttoc{toc}
  \makeatother

  \begin{lecture}{1}{The Class prBPP}{January 23, 2020}
\label{lec:01}

\subsection*{The Upshot}

\begin{enumerate}
  \item Randomness is necessary in many computational settings. Is randomness
    necessary for polynomial-time computation? There are good reasons to expect
    it is not! This is a fundamental question we will investigate in this
    course.
  \item Promise problems are a natural generalization of languages, and the
    theory of probabilistic algorithms for promise problems is richer than
    for that of languages.
  \item The class $\prBPP$ has a complete problem ($\CAPP$) under deterministic
    polynomial-time reductions. Hence to prove $\prBPP = \prP$ it suffices
    to give a deterministic polynomial-time algorithm for $\CAPP$.
  \item The class $\prBPP$ admits a time-hierarchy theorem.
  \item Probabilistic search problems can be reduced to probabilistic decision
    problems, meaning the theory we will develop around the class of decision
    problems $\prBPP$ will apply to the search analog as well.
\end{enumerate}

\subsection{Promise Problems and prBPP}

A \emph{promise problem} is a pair $\Pi = (\Pi_Y, \Pi_N) \subseteq \{0, 1\}^*
\times \{0, 1\}^*$ such that $\Pi_Y \cap \Pi_N = \emptyset$. The set $\Pi_Y$
captures all of the ``yes'' instances, and the set $\Pi_N$ captures all of the
``no'' instances. The set $\Pi_Y \cup \Pi_N$ is called the \emph{promise}. We
say that a Turing machine $M$ \emph{solves} a promise problem $\Pi$ if on
input $x \in Pi_Y$ it accepts and on input $x \in \Pi_N$ it rejects. Crucially,
for inputs not in the promise, the machine can behave arbitrarily.

Promise problems generalize decision problems (languages) in the case that
$\Pi_Y \cup \Pi_N = \Pi$. Complexity classes based on decision problems have
obvious analogs based on promise problems. For example, $\prP$ is the class of
all promise problems solvable by a deterministic polynomial-time Turing
machine. Similarly, the class $\prBPP$ is the set of decision problems solvable
by a probabilistic polynomial-time Turing machine with error probability at
most 1/3 for all instances. The class $\prBPP$ will be one of our main objects
of study, so its definition is worth repeating carefully.

\begin{definition}[$\prBPP$]
  A promise problem $\Pi = (\Pi_Y, \Pi_N)$ is in $\prBPP$ if and only if there
  exists polynomial $p(n)$ and a probabilistic Turing machine $M$ such that
  \begin{enumerate}
    \item if $x \in \Pi_Y$, then $M$ accepts input $x$ with probability at
      least $2/3$ after $p(n)$ steps, and
    \item if $x \in \Pi_N$, then $M$ rejects input $x$ with probability at
      least $2/3$ after $p(n)$ steps.
  \end{enumerate}
\end{definition}

The constant $2/3$ in the definition of $\prBPP$ is not consequential; it
can be replaced by any other constant greater than 1/2, which we will discuss
in more detail in the next section.%
\footnote{However, if the success probability is just ``strictly greater than
  1/2,'' then we obtain the class \PP, which contains, in particular, \NP and
  therefore is believed to be a strict superset of \BPP.}

\begin{proposition}
  If $\prBPP = \prP$, then $\BPP = \P$.
\end{proposition}

\begin{proof}
  If $\Pi \in \BPP$, then trivially $\Pi \in \prBPP$ and therefore by
  assumpiton $\Pi \in \prP$. As $\Pi$ is a decision problem, it follows
  that $\Pi \in \P$.
\end{proof}

What do we know about the placement $\BPP$ with respect to other complexity
classes? A trivial inclusion is $\BPP \subseteq \PSPACE \subseteq \EXP
\subseteq \NEXP$. Open problem: prove $\BPP \ne \NEXP$.

Using the amplification techniques of the next section, one can also prove
the following theorem.

\begin{theorem}
  $\BPP \subseteq \Ppoly$.
\end{theorem}


\subsection{Equivalent Definitions of prBPP via Amplification}

\begin{definition}
  An \emph{$(\eps, \delta)$-sampler} is a function $s :
  \set{0,1}^{\overline{T}} \times \set{0,1}^\ell \to \set{0,1}^T$ such that for
  all $f : \set{0,1}^T \to \set{0,1}$ and for all $z \in
  \set{0,1}^{\overline{T}}$ it holds that \[
    |\Exp_i f(s(z, i)) - \Exp_x f(x)| \le \eps.
  \]
\end{definition}

\begin{theorem}
  For all $\eps > 0$ there exists a polynomial-time computable $(\eps,
  \delta)$-sampler with $\overline{T} = \poly(T)$, $\ell = O(\log
  \overline{T}/\eps)$ and $\delta = 2^{\overline{T}^\eps - \overline{T}}$.
\end{theorem}

\begin{corollary}
  The class $\prBPP$ can be equivalently defined with an error of $2^{T^\eps -
  T}$ instead of 1/3, where $T = T(n)$ is the number of random bits.
\end{corollary}

\subsection{The Circuit Acceptance Probability Problem (CAPP)}

The \emph{acceptance probability} of a circuit is the fraction of inputs for
which it outputs 1.

\begin{definition}[$\CAPP$]
  $\CAPP$ is the promise problem where
  \begin{enumerate}
    \item $\CAPP_Y$ is the set of all descriptions of circuits that have
      acceptance probability at least 2/3, and
    \item $\CAPP_N$ is the set of all descriptions of circuits that have
      acceptance probability at most 1/3.
  \end{enumerate}
\end{definition}

\begin{theorem}[$\CAPP$ is $\prBPP$-complete]
  $\CAPP$ is complete for $\prBPP$ under deterministic polynomial-time reductions.
\end{theorem}

\comment{State the form of the Cook-Levin theorem we need in the following proof.}

\begin{proof}
  The problem $\CAPP$ is easily seen to be in $\prBPP$: consider the algorithm
  that simply simulate the input circuit on a randomly chosen input.

  Now we argue that every problem in $\prBPP$ can be reduced to $\CAPP$ in
  polynomial-time. Let $\Pi \in \prBPP$ and let $M$ be the probabilistic
  polynomial-time Turing machine solving $\Pi$. {\`A} la Cook-Levin, for every
  input $x$, we can construct in polynomial-time a circuit $C_x$ such that
  $C_x(r) = M(x, r)$. Hence if $x \in \Pi_Y$, then $C_x \in \CAPP_Y$, and if $x
  \in \Pi_N$, then $C_x \in \CAPP_N$.
\end{proof}

\subsection{A Time-Hierarchy Theorem}

\begin{theorem}
  $\prBPTIME[T \log{T}] \subsetneq \prBPTIME[T]$
\end{theorem}

\subsection{Reducing Search Problems to Decision Problems}

\subsection{Two-Sided Error vs.\ One-Sided Error}

\end{lecture}

  \begin{lecture}{2}{Hardness vs.\ Randomness}{January 27, 2022}

\subsection*{The Upshot}

\begin{enumerate}
  \item Hardness and randomness are tightly connected: hard functions can be
    used to design PRGs that stretch seeds of logarithmic length and fool small
    circuits. The quintisential construction of such a PRG is the
    Nisan-Wigderson (NW) PRG.
  \item The NW-PRG applies a hard function $f$ to many mostly disjoint
    projections (or subsequences) of the seed. The set of projections is given
    by a combinatorial design thats computable in polynomial time.
  \item The proof that the NW-PRG with a hard function $f$ yields a suitable
    PRG goes by contrapositive: if the PRG did not fool some circuit $D$, then we
    could \emph{reconstruct} from $D$ another small circuit that has small but
    significant agreement with $f$.
  \item A key technique in the above proof is the \emph{hybrid argument}, which
    states that if a circuit $D$ distinguishes the uniform distribution from
    another product distribution, then it distinguishes two ``hybrid'' product
    distributions that differ in a single coordinate.
\end{enumerate}

\subsection{Pseudorandomness}

\subsection{Hitting Set Generators}

\subsection{Pseudorandom Generators}

\subsection{The Nisan-Wigderson Generator}

\end{lecture}

  \begin{lecture}{3}{Error Correcting Codes and Hardness Amplification}{February 3, 2022}
\label{lec:03}

\subsection*{The Upshot}

\begin{enumerate}
  \item We have seen that if functions in \E are hard (on average) for circuits
    of size $2^{\eps n}$, then $\prBPP = \prP$. We can weaken the hypothesis
    signficantly and obtain the same conclusion: it suffices that there are
    no circuits of size $2^{\eps n}$ that compute $\E$.
\end{enumerate}
  

\subsection{A Primer on Coding Theory}

\subsection{Hardness Amplification via Locally List Decodable Codes}

\subsection{Reed-Solomon Codes}

\subsection{Reed-Muller Codes}

\end{lecture}

  \begin{lecture}{4}{Uniform Hardness vs.\ Randomness}{February 10, 2022}
\label{lec:04}

\subsection*{The Upshot}

\begin{enumerate}
  \item Uniform distinguishers for the NW-PRG with hard function $f$ can be
    used to learn $f$ with $1/\poly(n)$ advantage with high probability.
  \item The small advantage can be used to bootstrap an algorithm that computes
    a well-structured problem in $\P^{\sharpp}$ with advantage. The conclusion is
    that if $\prBPP = \prP$ unless there are algorithms to solve problems in
    $\P^{\sharpp}$ on average. \comment{Is this right? I think we are missing
    a concrete hardness claim about the problem.}
  \item LLDCs and downward self-reducibility play a crucial role in the
    bootstrapping algorithm.
\end{enumerate}


\subsection{Uniform Circuits}

So far we have been studying PRGs for non-uniform circuits. As a consequence,
our results so far require problems in \E that are hard for non-uniform models.
In this lecture we investigate the analgous questions for uniform models of
computation.

\comment{Define uniform circuits}

\begin{definition}[$\eps$-PRG for uniform circuits]
  An algorithm $G$ is an \emph{$\eps$-PRG for uniform circuits generated in
  time $t(n)$} if for every Turing machine $D$ with running time $t(n)$ and
  large enough $n \in \N$
	\[
		\Pr\brac{D(1^n) \text{ is an $\eps$-distinguisher for } G(1^n, u_{l(n)})} \leq \eps.
	\]
\end{definition}

\begin{definition}
  Let $\overline{x} = \set{\overline{x}_n}$ be a sequence of distributions over
  $\set{0, 1}^n$ and let $p : \N \to \N$. We say that $\overline{x}$ is
  \emph{$p$-time samplable} if there is a probabilistic machine $S$ with
  runtime $p(n)$ such that $S(1^n)$ has the distribution $\overline{x}(n)$.
\end{definition}

\begin{theorem}
  Assume that for all polynomials $p(n)$ there is an $\eps$-PRG $G$ for uniform
  circuits generated in time $p(n)$ that stretches a $\log(n)$-sized seed. Then
  for all $L \in \BPP$ and poly-time samplable distributions $\overline{x}$,%
  \footnote{Think of $\overline{x}$ as an adversary (limited to polynomial
  time) trying to generate hard inputs with $\geq \eps$ probability.} there
  exists a language $L' \in \textsf{P}$ such that
	\[
		\Pr_{x \sim \overline{x}}\brac{L(x) \neq L'(x)} \leq \eps.
	\]
\end{theorem}

\begin{proof}
	Let $S$ be the sampling algorithm for $\overline{x}$, and let $M = M_L$
	be the $\BPP$ machine for $L$. We define $L'$ as: given $x$, enumerate
	over seeds $s$ of $G$, and output
	\[ \textsf{MAJ}_{s} \set{M(x, G(s)) }. \]
	Since the seed length $s$ is $O(\log n)$, the algorithm above runs in
	polynomial time, and hence $L' \in \P$.
	Further, for each $x$, we define:
	\[
		\Delta(x) = \card{
			\Pr_{r \sim \overline{u}_n}         \brac{ M(x, r) = 1}
		  - \Pr_{s \sim \overline{u}_{\ell(n)}} \brac{ M(x, s) = 1}
		  }.
	\]

	We suppose (towards a contradiction) that $L'$ differs from $L$ on more
	than an $\eps$-fraction of inputs. Since $M$ is a $\BPP$ machine for $L$
	this means that
	$\Pr_{x \sim \overline{x}_n}\brac{\Delta(x) > \eps} > \eps$. Then consider
	the algorithm $D$: on the input $1^n$, sample $x \sim \overline{x}_n$ using
	$S(1^n)$, and then output the circuit $D_x(r) = M(x, r)$.
	Then
  \begin{align*}
		\Pr\brac{D \text{ prints a circuit that } \eps \text{-distinguishes } G(1^n, u_{\ell})}
    &= \Pr\brac{\Delta(S(1^n)) > \eps}\\
    &= \Pr_{x \sim \overline{x}_n}\brac{\Delta(x) > \eps}\\
    &> \eps,
  \end{align*}
	which contradicts the fact that $G$ is an $\eps$-PRG.
\end{proof}

\begin{theorem}
	If
	$\SPACE\brac{O(n)}$ is hard for $\BPTIME\brac{2^{\eps n}}$
	then
	$\RP = \P$ ``on average''.
\end{theorem}

\newcommand{\fws}{\ensuremath{ f^{\text{ws}} }}
\begin{theorem}
	There exists a language $\fws \in \P^{\sharpp}$ such that if $\fws$
	is hard for $\BPTIME\brac{2^{\eps n}}$ then
	$\BPP = \P$ ``on average''.
\end{theorem}

\begin{proof}
	\comment{TODO.}
\end{proof}

\subsection{Learning via Uniform Distinguishers}

\comment{Restate / refer to the Nisan-Wigderson PRG here.}

\begin{definition}
  Let $f_n : \set{0, 1}^n \to \set{0, 1}.$ A probabilistic algorithm $A$
  \emph{learns} with oracle access to $f_n$ \emph{learns $f_n$ with success
  $\delta$ and advantage $\eps$} if on input $1^n$, $A$ produces a circuit
  $C_n$ such that $\Pr_x \brac{C_n(x) = f_n(x)} \geq 1/2 + \eps$.
\end{definition}

\begin{proposition}\label{prop:learning}
  Let $\ell = \log{n}$, let $\gamma \in (0, 1/2)$, and let $T_1, \dots, T_n$ be
  a $(\ell, \gamma \ell)$-design over $[O(\ell)]$. Finally, let $f :
  \set{0,1}^\ell \to \set{0, 1}$ and let $G^f$ be the Nisan-Wigderson PRG with
  oracle $f$ and design $T_1, \dots, T_n$. If $D$ is a uniform circuit on $n$
  inputs generated in time $\poly(n)$ that is not $(1/n)$-fooled by $G^f$, then
  there is a $\poly(n)$-time algorithm to learn $f$ with advantage $1/n^2$ and
  success $1/n$.
\end{proposition}

\begin{proof}
  By definition, if $D$ is not $(1/n)$-fooled by $G^f$, then
	\[
		\card{\Pr\brac{D(G(1^n, s)) = 1} - \Pr\brac{D(\overline{u}_n) = 1}} > 1/n.
	\]
  \comment{Should the input to $G$ really be $1^n$ above?}

  As in the hybrid argument (\comment{ref. lecture 2 here}), we define the
  distributions
  \begin{align*}
    \overline{H}_0 &= (f(x_1), \ldots, f(x_n)),\\
    \overline{H}_i &= (\overline{u}_i, f(x_{i + 1}), \ldots, f(x_n)),\\
    \overline{H}_n &= \overline{u}_n,
  \end{align*}
  where $x_i$ is the projection of the seed given to the PRG onto the
  coordinates $T_i$. Recall the (easy) lemma that there is an $i$ such that:
	\[
		\card{\Pr\brac{D(\overline{H}_{i - 1}) = 1} - \Pr\brac{D(\overline{H}_i) = 1}} > 1 / n^2.
	\]
  Roughly speaking, the learning algorithm will guess the value of this $i$ and
  exploit the fact that $D$ can differentiate the bit $f(x_i)$ from a uniformly
  random bit. More concretely, we describe the algorithm $A$:
	\begin{enumerate}
    \item Construct the a $(\ell, \gamma \ell)$-design $T_1, \dots, T_n$ in
      time $\poly(n)$ using Theorem~\ref{thm:design} from
      Lecture~\ref{lec:02}.
		\item Choose a random $i \in \brac{n}$.
		\item Choose a random $z^* \in \set{0, 1}^{\ell(n) - \ell}$. Here $s$
			(the seed of the PRG) projected to $T_i$ will be equal to $x_i$,
			and the remaining coordinates will be filled by $z^*$.
		\item Choose a random bit $\sigma$.
		\item Query the oracle for $f$ on all possible inputs
			$x_{i + 1}, \ldots , x_n$. Note that at most $\ell / 100$ bits of
			each of these $x_j$'s are unknown after fixing $z^*$, so there
			are at most $n \cdot 2^{\ell / 100}$ queries to the oracle.
		\item Choose a random $u^* \in \set{0, 1}^{i - 1}$.
		\item Output the circuit $C_n$ which simulates $D$ on
			$(u^*, \sigma, f(x_{i + 1}), \ldots , f(x_n))$ and outputs
			$\sigma$ if $D$ accepts, and $\lnot \sigma$ otherwise.
	\end{enumerate}
	It is easy\footnote{haha} to see that
	$\Pr\brac{C_n(x) = f_n(x)} = 1/2 + 1/n^2$.
\end{proof}

\begin{corollary}
  If there is a $(1/n)$-distinguisher for the NW-PRG $G^f$, then there is an
  algorithm that learns $f$ with success $1 - 1/\delta$ and advantage
  $1/(2n^2)$ that runs in time $\poly(n, 1/\delta)$.
\end{corollary}

\begin{proof}
  Let $A$ be the learning algorithm for $f$ with success $1/n$ and advantage
  $1/n^2$ promised by Proposition \ref{prop:learning}. By running $A$
  independently $n \log(1/\delta)$ times, the probability that none of the
  circuits produced has advantage $1/n^2$ is $1 - (1 - 1/n)^{n\log(1/\delta)} >
  1 - \delta$. We can estimate the acceptance probability of each of the
  $n\log(1/\delta)$ circuits to within an additive error of $1/(2n^2)$ with
  probability $1 - \delta$ by running it on $O(n^4 \log(1/\delta))$ random
  inputs. Hence if such a circuit exists, we will find it in time $\poly(n,
  1/\delta)$. Reparameterizing $\delta$ gives the result.
\end{proof}

\subsection{Bootstrapping via Downward Self-Reducibility and LLDCs}

\begin{definition}
  $f$ is \emph{downward self-reducable (DSR)} if there exists an algorithm $A$
  such that when $A$ gets $x \in \set{0, 1}^n$ and oracle access to $f$ on $n -
  1$ bits, it outputs $f(x)$ in linear time.%
  \footnote{Linear time is not critical, it can be replaced (for example) by quadratic time ... but not $\poly$-time (???).}
\end{definition}

For example, $\SAT$ and $\PERM$ are DSR.

\begin{proposition}
	There exists a function $\fws \in \P^{\sharpp}$ such that:
	\begin{enumerate}
		\item $\fws$ is DSR.
		\item For all $\ell \in \N$ the truth table of $\fws$ is a codeword
			which is:
			\begin{enumerate}
        \item Locally List Decodable from agreement $\eps(\ell) = 2^{-\ell /
          100}$ in time $\poly(1/\eps)$ with list size $\poly(1 / \eps)$.
        \item Uniquely Decodable from agreement $0.99$ in time $\poly(n)$.
			\end{enumerate}
	\end{enumerate}
\end{proposition}

\begin{theorem}
  If $\fws$ is hard for $\BPTIME\brac{2^{\eps n}}$ then for all polynomials
  $p(n)$ there is a $(1 / n)$-PRG for uniform circuits generated in time $p(n)$
  with $\log$-sized seed and polynomial running time.
\end{theorem}

\begin{proof}
	\comment{TODO.}
\end{proof}

\end{lecture}


\end{document}
