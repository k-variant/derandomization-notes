\begin{lecture}{2}{Hardness vs.\ Randomness}{January 27, 2022}
\label{lec:02}

\subsection*{The Upshot}

\begin{enumerate}
  \item Hardness and randomness are tightly connected: hard functions can be
    used to design PRGs that stretch seeds of logarithmic length and fool small
    circuits. The quintisential construction of such a PRG is the
    Nisan-Wigderson (NW) PRG.
  \item The NW-PRG applies a hard function $f$ to many mostly disjoint
    projections (or subsequences) of the seed. The set of projections is given
    by a combinatorial design thats computable in polynomial time.
  \item The proof that the NW-PRG with a hard function $f$ yields a suitable
    PRG goes by contrapositive: if the PRG did not fool some circuit $D$, then we
    could \emph{reconstruct} from $D$ another small circuit that has small but
    significant agreement with $f$.
  \item A key technique in the above proof is the \emph{hybrid argument}, which
    states that if a circuit $D$ distinguishes the uniform distribution from
    another product distribution, then it distinguishes two ``hybrid'' product
    distributions that differ in a single coordinate.
\end{enumerate}

\subsection{Pseudorandomness}

\subsection{Hitting Set Generators}

\subsection{Pseudorandom Generators}

\subsection{The Nisan-Wigderson Generator}

\begin{definition}[Combinatorial design]
  A collection of subsets $T_1, \dots, T_m \subseteq [d]$ is a \emph{$(\ell, a)$-design}
	\begin{enumerate}
    \item $\card{T_i} = \ell$ for all $i$, and
		\item $\card{T_i \cap T_j} < a$ for all $i \ne j$.
	\end{enumerate}
\end{definition}

We will be interested in combinatorial designs where $m = 2^\ell$, $d =
O(\ell)$, and $a = \gamma \ell$ for some constant $\gamma > 0$. The following
proposition states that, not only does such a design exist for all $\ell$,
but it can be computed deterministically in time $O(2^\ell)$.

\begin{theorem}\label{thm:design}
  Let $\gamma > 0 $ and let $\ell, m \in \mathbb{N}$. For all $a \ge \gamma
  \log{m}$ and $d \ge e^2 \cdot 2^{1/\gamma} \cdot \ell^2/a$, there exists an
  $(\ell, a)$-design $T_1, \dots, T_m \subseteq [d]$. Moreover, such a design
  can be found deterministically in time $\poly(m, d)$.
\end{theorem}

The proof of the theorem will use the following lemma.
\begin{lemma}\label{lem:design}
  Let $a, \ell, d \in \N$ with $a \le \ell \le d$, and let $T_1, \dots, T_m \in
  \binom{[d]}{\ell}$. If $m < \binom{d}{a} / \binom{\ell}{a}^2$, then there
  exists a set $T^* \in \binom{[d]}{\ell}$ such that $\card{T^* \cap T_i} < a$
  for all $i \in [m]$. Moreover, such a set $T^*$ can be found
  deterministically in time $\poly(m, d)$.
\end{lemma}

\begin{proof}
  We first use the probabilistic method to show that such a set $T^*$ exists.
  Let $\mathbb{T}$ be a uniformly random $\ell$-set over $[d]$, let $X_j$
  be an indicator for the ``bad'' event $B_j$ that $|\mathbb{T} \cap T_j| \ge
  a$, and let $X = \sum_{j=1}^m X$ be the number of bad events. It suffices to
  show that \[
    \Exp\brac*{X} < 1,
  \]
  because then there is some realization of $\mathbb{T}$ for which all $X_j$
  are 0.

  The number of $\ell$-sets whose intersection with $T_j$ is at least $a$ is
  most $\binom{\ell}{a}\binom{d - a}{\ell - a}$: first choose $a$ elements from
  $T_j$ and then pick any $\ell - a$ elements from the remaining elements. Hence
  \[
    \Pr(B_j) \le \frac{\binom{\ell}{a}\binom{d - a}{\ell - a}}{\binom{d}{\ell}} =
    \frac{\binom{\ell}{a}^2}{\binom{d}{a}},
  \]
  where we used the identity $\binom{d}{\ell}\binom{\ell}{a} =
  \binom{d}{a}\binom{d - a}{\ell - a}$ in the last step. If $m < \binom{d}{a} /
  \binom{\ell}{a}^2$, then \[
    \Exp\brac*{X} =
    \sum_{j=1}^m \Pr(B_j) \le m \cdot \frac{\binom{\ell}{a}^2}{\binom{d}{a}} < 1,
  \]
  as desired.

  Now we describe how to construct such a set $T^*$ deterministically using the
  \emph{method of conditional probabilities}. At a high level, the algorithm
  simply iterates over the elements $i \in [d]$ and in each iteration decides
  whether to include $i$ in $T^*$ based on a conditional expectation
  calculation. For example, suppose we want to decide if the element $1$ should
  be included. We have \[
    1 > \Exp\brac*{X} = \Pr(1 \in \mathbb{T}) \cdot \Exp\brac*{X \mid 1 \in \mathbb{T}} 
      + \Pr(1 \not\in \mathbb{T}) \cdot \Exp\brac*{X \mid 1 \not\in \mathbb{T}},
  \]
  and so one of $\Exp\brac{X \mid 1 \in \mathbb{T}}$ or $\Exp\brac{X \mid 1 \not\in
  \mathbb{T}}$ is less than one. If $\Exp\brac{X \mid 1 \in \mathbb{T}} < 1$,
  then it is safe to add $1$ to $T^*$. Inductively, when we consider adding
  element $i$ to $\mathbb{T}$, one of the two expectations will be less than
  one, which gives us a decision for $i$.

  The only thing left to show is that for any set $T \subseteq [i]$, we can
  compute $\Pr(B_j \mid \mathbb{T} \cap [i] = T)$ exactly. This is a bit of an
  ugly calculation, but only involves basic combinatorics; we leave the details
  to the reader.
\end{proof}
%    \Pr(B_j \mid \mathbb{T} \cap [i] = T) = \binom{d - i}{\ell - |T|}^{-1}
%    \sum_{b=a}^{\ell} \binom{|S_j \cap \set{i+1, \dots, d}|}{b-|T \cap S_j|}\binom{d - \ell}{\ell}

Now we return to the proof of Theorem~\ref{thm:design}.

\begin{proof}[Proof of Theorem~\ref{thm:design} using Lemma~\ref{lem:design}]
  A routine calculation shows that for $a \ge \gamma \log{m}$ and $d \ge e^2
  \cdot 2^{1/\gamma} \cdot \ell^2/a$, we have $m \le \binom{d}{a} /
  \binom{\ell}{a}^2$. Indeed,
  \begin{align*}
    \binom{d}{a} \cdot \binom{\ell}{a}^{-2}
    &\ge \paren*{\frac{d}{a}}^a \cdot \paren*{\frac{e \ell}{a}}^{-2a}\\
    &\ge \paren*{\frac{e^2 \cdot 2^{1/\gamma} \cdot \ell^2}{a^2}}^{a} \cdot \paren*{\frac{e^2 \ell^2}{a^2}}^{-a}\\
    &= 2^{a / \gamma}\\
    &\ge m.
  \end{align*}
  Hence we can iteratively apply Lemma~\ref{lem:design} $m$ times to obtain the
  sets $T_1, \dots T_m$.
\end{proof}

\end{lecture}
