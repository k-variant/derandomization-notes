\begin{lecture}{2}{Hardness vs.\ Randomness}{January 27, 2022}
\label{lec:02}

\subsection*{The Upshot}

\begin{enumerate}
  \item Hardness and randomness are tightly connected: hard functions can be
    used to design PRGs that stretch seeds of logarithmic length and fool small
    circuits. The quintisential construction of such a PRG is the
    Nisan-Wigderson (NW) PRG.
  \item The NW-PRG applies a hard function $f$ to many mostly disjoint
    projections (or subsequences) of the seed. The set of projections is given
    by a combinatorial design thats computable in polynomial time.
  \item The proof that the NW-PRG with a hard function $f$ yields a suitable
    PRG goes by contrapositive: if the PRG did not fool some circuit $D$, then we
    could \emph{reconstruct} from $D$ another small circuit that has small but
    significant agreement with $f$.
  \item A key technique in the above proof is the \emph{hybrid argument}, which
    states that if a circuit $D$ distinguishes the uniform distribution from
    another product distribution, then it distinguishes two ``hybrid'' product
    distributions that differ in a single coordinate.
\end{enumerate}

\subsection{Pseudorandomness}

\subsection{Hitting Set Generators}

\subsection{Pseudorandom Generators}

\subsection{The Nisan-Wigderson Generator}

\begin{definition}[Combinatorial design]
  A collection of subsets $T_1, \dots, T_m \subseteq [d]$ is a \emph{$(\ell, a)$-design}
	\begin{enumerate}
    \item $\card{S_i} = \ell$ for all $i$, and
		\item $\card{S_i \cap S_j} < a$ for all $i \ne j$.
	\end{enumerate}
\end{definition}

\begin{proposition}\label{prop:design}
  Let $\gamma > 0 $ and let $\ell, m \in \mathbb{N}$. For all $a \ge \gamma
  \log{n}$ and $d \ge e^2 \cdot 2^{1/\gamma} \cdot \ell^2/a$, there exists an
  $(\ell, a)$-design $T_1, \dots, T_m \subseteq [d]$. Moreover, such a design
  can be found deterministically in time $\poly(m, d)$.
\end{proposition}

We will be interested in the parameter regime where $\gamma$ is a small
constant (say, $1/10$) and $m = 2^\ell$. In this regime, the set intersection
sizes are bounded by $\gamma \ell$ (no two sets intersect in more than a
constant fraction of their elements) and the universe size $d$ is $O(\ell)$.

\comment{Add proof. The sketch is given in Problem 3.2 of Vadhan's pseudorandomness.}

\end{lecture}
