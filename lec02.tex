\begin{lecture}{2}{Hardness vs.\ Randomness}{January 27, 2022}
\label{lec:02}

\subsection*{The Upshot}

\begin{enumerate}
  \item Hardness and randomness are tightly connected: hard functions can be
    used to design PRGs that stretch seeds of logarithmic length and fool small
    circuits. The quintisential construction of such a PRG is the
    Nisan-Wigderson (NW) PRG.
  \item The NW-PRG applies a hard function $f$ to many mostly disjoint
    projections (or subsequences) of the seed. The set of projections is given
    by a combinatorial design thats computable in polynomial time.
  \item The proof that the NW-PRG with a hard function $f$ yields a suitable
    PRG goes by contrapositive: if the PRG did not fool some circuit $D$, then we
    could \emph{reconstruct} from $D$ another small circuit that has small but
    significant agreement with $f$.
  \item A key technique in the above proof is the \emph{hybrid argument}, which
    states that if a circuit $D$ distinguishes the uniform distribution from
    another product distribution, then it distinguishes two ``hybrid'' product
    distributions that differ in a single coordinate.
\end{enumerate}

\subsection{Pseudorandomness}

\subsection{Hitting Set Generators}

\subsection{Pseudorandom Generators}

\subsection{The Nisan-Wigderson Generator}

\end{lecture}
