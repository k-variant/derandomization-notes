%! TeX root = main.tex

\begin{lecture}{10}{Non-black-box derandomization using doubly-efficient proof
  systems}{March 31, 2022}\label{lec:non-black-box}

\subsection*{The Upshot}

\begin{enumerate}
  \item We have seen equivalences between non-uniform hardness results and the
    existence of PRGs that beat circuits (\cref{lec:02}), and between uniform
    hardness results and PRGs that beat uniform circuits (\cref{lec:04}).
    These results gave us worst case and average case derandomization
    respectively, by simply using the PRG to generate the randomness for a
    $\BPP$ algorithm.
  \item However, the derandomization above is blackbox --- the output of the
    PRG depends only on the seed (and not the input to the algorithm we are
    trying to derandomize).
  \item By instead designing \emph{targeted} PRGs, we will get closer to an
    equivalence between the hardness assumption and derandomizing $\BPP$.
\end{enumerate}

\subsection{The Plan}
TODO

\subsection{Preliminaries}
TODO

\subsection{Bootstrapping System via Double Efficient Proofs}
TODO

\subsection{From Bootstrapping Systems to Targeted HSGs}
TODO

\end{lecture}
