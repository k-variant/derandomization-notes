%! TeX root = main.tex

\begin{lecture}{10}{Non-black-box derandomization using doubly-efficient proof
  systems}{March 31, 2022}\label{lec:non-black-box}

\subsection*{The Upshot}

\begin{enumerate}
  \item We have seen equivalences between non-uniform hardness results and the
    existence of PRGs that beat circuits (\cref{lec:02}), and between uniform
    hardness results and PRGs that beat uniform circuits (\cref{lec:04}).
    These results gave us worst case and average case derandomization
    respectively, by simply using the PRG to generate the randomness for a
    $\BPP$ algorithm.
  \item However, the derandomization above is blackbox --- the output of the
    PRG depends only on the seed (and not the input to the algorithm we are
    trying to derandomize).
  \item By instead designing \emph{targeted} PRGs, we will get closer to an
    equivalence between the hardness assumption and derandomizing $\BPP$.
\end{enumerate}

\subsection{The Plan}
TODO

\subsection{Preliminaries}

\begin{definition}[$\eps$-targeted PRG]%
  \label{defn:targeted-PRG}
  For $T : \N \to \N$, let $G$ be an algorithm that gets input:
  \begin{itemize}
    \item $x \in \set{0, 1}^n$ and
    \item a seed of length $\ell(n)$,
  \end{itemize}
  and outputs a string of length $T(n)$.
  We say $G$ is an $\eps$-targeted PRG for time $T$ if for every algorithm
  $A$ running in time $T$, $n$ large enough, and $x \in \set{0, 1}^n$ it holds
  that
  \[
    \card*{ \Pr_{r \in \set{0, 1}^{T(n)}} \brac{ A(x, r) = 1 } -
    \Pr_{s \in \set{0, 1}^{\ell(n)}} \brac{ A(x, G(x, s)) = 1 }} \leq \eps.
  \]
\end{definition}
Note that the only difference between this definition and that of a PRG is
the additional input of $x$ to $G$.

TODO

\subsection{Bootstrapping System via Double Efficient Proofs}
TODO

\subsection{From Bootstrapping Systems to Targeted HSGs}
TODO

\end{lecture}
